
\documentclass[11pt, oneside]{article}

%% Packages
\usepackage{geometry}
\geometry{letterpaper}
\usepackage{changepage}   % for the adjustwidth environment
\usepackage{graphicx}
\usepackage{wrapfig}
\graphicspath{ {images/} }
\usepackage{amssymb}
\usepackage{amsmath}
\usepackage{amscd}
\usepackage{hyperref}
\hypersetup{
    colorlinks=true,
    linkcolor=blue,
    filecolor=magenta,
    urlcolor=blue,
}
\usepackage{xcolor}
\usepackage{soul}


%% Commands
\newcommand{\code}[1]{{\tt #1}}
\newcommand{\ellie}[1]{\href{#1}{Link to Ellie}}
% \newcommand{\image}[3]{\includegraphics[width=3cm]{#1}}

\newcommand{\imagecenter}[3]{{
   \medskip
   \begin{figure}
   \centering
    \includegraphics[width=12cm,height=12cm,keepaspectratio]{#1}
    \vglue0pt \par {#2}
    \end{figure}
    \medskip
}}

\newcommand{\imagefloatright}[3]{
    \begin{wrapfigure}{R}{0.30\textwidth}
    \includegraphics[width=0.30\textwidth]{#1}
    \caption{#2}
    \end{wrapfigure}
}

\newcommand{\imagefloatleft}[3]{
    \begin{wrapfigure}{L}{0.3-\textwidth}
    \includegraphics[width=0.30\textwidth]{#1}
    \caption{#2}
    \end{wrapfigure}
}

\newcommand{\italic}[1]{{\sl #1}}
\newcommand{\strong}[1]{{\bf #1}}
\newcommand{\subheading}[1]{{\bf #1}\par}
\newcommand{\xlinkPublic}[2]{\href{{http://www.knode.io/\#@public#1}}{#2}}
\newcommand{\red}[1]{\textcolor{red}{#1}}
\newcommand{\blue}[1]{\textcolor{blue}{#1}}
\newcommand{\remote}[1]{\textcolor{red}{#1}}
\newcommand{\local}[1]{\textcolor{blue}{#1}}
\newcommand{\highlight}[1]{\hl{#1}}
%% \newcommand{\meta}[2]{\textcolor{blue}{#1}: }{\hl{#1}}
\newcommand{\strike}[1]{\st{#1}}
\newcommand{\term}[1]{{\sl #1}}
\newtheorem{remark}{Remark}
\newcommand{\comment}[1]{}
\newcommand{\innertableofcontents}{}

%% Theorems
\newtheorem{theorem}{Theorem}
\newtheorem{axiom}{Axiom}
\newtheorem{lemma}{Lemma}
\newtheorem{proposition}{Proposition}
\newtheorem{corollary}{Corollary}
\newtheorem{definition}{Definition}
\newtheorem{example}{Example}
\newtheorem{exercise}{Exercise}
\newtheorem{problem}{Problem}
\newtheorem{exercises}{Exercises}
\newcommand{\bs}[1]{$\backslash$#1}
\newcommand{\texarg}[1]{\{#1\}}

%% Environments
\renewenvironment{quotation}
  {\begin{adjustwidth}{2cm}{} \footnotesize}
  {\end{adjustwidth}}

% Spacing
\parindent0pt
\parskip5pt

\begin{document}




 \title{Note  on  the  Lambda  Calculus}
 \maketitle
We  consider  some  computations  in  the  lambda-calculus.  These  are  function  applications,  some  which  have  reduction  sequences  that  terminate  in  a  finite  number  of  steps,  others  that  do  not.   Possibile  non-termination  is  a  feature  of  the   \italic{untyped} "original"  lambda  calculus.   Later  Church  introduced  the  simply-typed  lambda  calculus,  in  which  such  behavior  is  excluded.


 \subheading{Example  1}
Consider  the  lambda  expression   $\alpha = \lambda x. x + x$  applied  to  the  integer  2.  This  function  application  reduces  to  the  expression   $2 + 2$  which  in  turn  reduces  to  the  integer   $4$ .   No  further  reduction  is  possible.   The  reduction  process  terminates  in  a  finite  number  of  steps.  We  write  this  as


$$
 (\lambda x. x + x) 2 \to 2 + 2 \to 4
$$

 \subheading{Example  2}
Let    $\phi = \lambda x. xx$  and  consider  the  expression   $\phi \phi$ .  It  "reduces"  as  follows:


\begin{align}
\phi \phi & = (\lambda x. xx)(\lambda x. xx) \\
  & = (\lambda x. xx)(\lambda x. xx)  \\
& = (\lambda x. xx)(\lambda x. xx)  \\
& etc.
\end{align}

The   reduction  sequence


$$
\phi \phi \to \phi\phi \to \phi\phi \to \cdots
$$

is  nonterminating.


 \subheading{Example  3}
Let    $\psi = \lambda x. xxx$  and  consider  the  expression   $\psi \psi$ :


\begin{align}
\psi \psi & = (\lambda x. xxx)(\lambda x. xxx) \\
  & = (\lambda x. xx)(\lambda x. xx) (\lambda x. xx) \\
& = (\lambda x. xx)(\lambda x. xx)(\lambda x. xx)(\lambda x. xx)  \\
& etc.
\end{align}

Again,  the  reduction  sequence


$$
\psi\psi \to \psi\psi\psi \to \psi\psi\psi\psi \to \cdots
$$

is  non-terminating.


 \subheading{Simply-typed  lambda  calculus.}
The  non-terminating  reduction  sequences  observed  above  arose  when  certain  function  were  applied  to  themselves.   Consider  now  a  typed  language,  so  that  functions  from  things  of  type   $A$  to  things  of  type   $B$  themselves  have  a  type,  which  one  writes  as   $A \to B$ .   Let   $f$  be  a  function  of  this  type.   Then  we  may  apply   $f$  to  things  of  type   $A$  but  not  to  itself,  since  the  argument   in   $f$  applied  to   $f$  has  type   $A \to B$ .




\end{document}
